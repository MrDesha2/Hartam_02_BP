\newpage
\subsection{Описание бизнес-процесса «Планирование производства»}
\label{bp:plan}


\begin{comment} %НАЧАЛО КОММЕНТАРИЯ
\subsubsection{Сценарий ''Заполнение мощностей рабочих центров для предварительного планирования''}
\label{bp:plan_1}


\begin{enumerate}

\item \planner для каждого рабочего центра предприятия заполняет доступные мощности в системе \gofro в регистре \myobject{ОбъемыРабочихЦентров} на каждую дату производственного календаря.
\item Если ожидается поступление срочных или VIP заказов, тогда \planner занижает (при необходимости) емкости рабочих центров.
\end{enumerate}


\ifnum\IsOKP=1    % % Нужно ли им ОКП???
\subsubsection{Сценарий ''Предварительное планирование производства''} 
\label{bp:plan_2}



\begin{enumerate}

\item \manager для определения возможности отгрузки продукции к затребованному заказчиком сроку в системе \gofro  в документе \myobject{Заявка} указывает позиции номенклатуры, желаемую дату и нажимает кнопку \myform{РасчетДаты}. \todo{Это откуда-то осталось. Будет ОКП у них???}
\item Модуль предварительного планирования из системы \gofro рассчитывает возможную дату и возвращает результат. Для каждой позиции строки документа \myobject{Заявка} определяется дата, в которую изделие можно отгрузить (с учетом объема уже набранных производственных заказов и заданной мощности рабочих центров). Если невозможно отгрузить в дату, указанную пользователем, то пользователю указывается другая предлагаемая системой \gofro дата отгрузки из горизонта поиска. Если изделие отгрузить в указанный период невозможно, то возвращается пустая дата. Если система \gofro предложила изменить дату отгрузки или отгрузка невозможна, то по каждому такому изделию дополнительно возвращается сообщение о станках, которые перегружены по объемам. 

\item	Предлагаемая системой \gofro дата служит ориентиром для \manager, но не являются ограничением на текущем этапе (ограничения появляются на этапе создания производственных заказов).
\item	\manager просматривает предложенные системой \gofro даты изготовления заказа, принимает их или указывает другую желаемую дату производства.
\item	При планировании неопределенной заказчиком продукции \manager планирует предварительные отгрузки в таблице \blue{@MSExcel}. \manager сообщает \planner объемы предварительных продаж по неопределенным позициям. 

\planner в системе \gofro уменьшает квоты производства по рабочим центрам на указанные даты на указанный менеджером объем с указанием причины уменьшения квот. 
\item	При появление внеочередных заказов и необходимости их срочного изготовления \manager должен сообщить \planner о необходимости увеличения квот.

\planner в системе \gofro увеличивает квоты производства по рабочим центрам на указанные даты на указанный менеджером объем. При этом система не контролирует какой из заказов займет квоту.

\item	Система \gofro при проведении документа \myobject{Заказ} резервирует производственные мощности под указанные производственные заказы для учета в дальнейшем предварительном планировании. При процедуре резервирования система \gofro автоматически заново рассчитывает даты загруженности по рабочим центрам. При положительном результате подбора даты для документа \myobject{Заказ} по рабочим центрам маршрута происходит резервирование соответствующих объемов в рабочих центрах на подобранные даты (если изделие требует нескольких шагов маршрута, то на резерве оно может стоять на разные даты с учетом итогового выпуска к дате отгрузки). Если для указанных параметров системе \gofro не удается подобрать даты до даты желаемой отгрузки (включительно), то на те шаги и рабочие центры, где удалось подобрать, резерв будет записан на подобранную дату, а на все остальные шаги на желаемую дату отгрузки.
\item	Для получения информации о текущих набранных объемах \manager в системе \gofro может использовать отчет \myform{ЗагруженностьРабочихЦентров}.


\end{enumerate}

\endif



\end{comment}


\subsubsection{Сценарий ''Планирование выпуска раскроев на гофроагрегате''}
\label{bp:plan_3}


\begin{enumerate}


\item Для выполнения нового планирования \planner создает в системе \gofro новый документ \myobject{План} в журнале планов. 
% \item	\planner для построения раскроев и выдачи задания на гофроагрегат  создает план. % с типом «Гофроагрегат».
%\item	\planner для загрузки в системе \gofro списка заказов, которые были запланированы в предыдущем плане, но по определенным причинам не были выпущены (отставание гофроагрегата или заведомо больший объем раскроев в предыдущем плане и т.п.),  в документе плана нажимает кнопку \$ЗагрузкаЗаказовИзДругогоПлана и выбирает предыдущий план по гофроагрегату. Система \gofro анализирует все запланированные в крой заказы в предыдущем плане и в текущий план загружает только те, по которым еще требуется выпуск заготовок на гофроагрегате. Потребность в заготовках рассчитывается на базе исходного объема заказа системы \gofro и объемов, прошедших на текущий момент в системе \gofro по документам выработки.
\item	\planner  для загрузки в системе \gofro списка новых одобренных \manager к выпуску заказов  в документе \myobject{План} нажимает кнопку \myform{ДозагрузкаЗаказов} и выбирает диапазон дат отгрузки, за какой период заказы он желает видеть у себя в плане. По каждому из активных документов \myobject{Заказ} системы \gofro, у которых дата желаемой отгрузки содержится в указанном диапазоне, система \gofro анализирует необходимый для кроя объем заготовок на ГА, и, если этот объем положительный, добавляет заказ в список доступных для планирования в рамках данного плана. Потребность в заготовках рассчитывается на базе исходного объема документа \myobject{Заказ} системы \gofro, ранее запланированных объемов и объемов проведенных на текущий момент в системе \gofro по документам выработки (\myobject{ВыработкаПоПереработке}, \myobject{ВыработкаГофроагрегата}). В план попадают проведенные документы \myobject{Заказ} со статусом «Одобрено к выпуску».
\item	При необходимости загрузки срочных заказов \planner может выполнить в системе \gofro дозагрузку в план одного или нескольких заказов и после контрольного времени, но данная ситуация должна быть дополнительно согласована между \manager  и \planner устно.
\item \planner в документе \myobject{План} в системе \gofro на вкладке «Список заказов для планирования»  отмечает те заказы, по которым желает построить раскрои и переходит на вкладку плана «Построение раскроев».
\item \planner в системе \gofro по команде \myform{Расчет} рассчитывает раскрои для всех включенных заказов, после чего \planner при необходимости корректирует их: изменяет порядок следования, объем выпуска и т.п. Раскрои, которые удовлетворяют требованиям, \planner отмечает как “сохраненные”, после чего в последующих расчетах внутри плана эти раскрои системой \gofro не изменяются.
\item	 \planner для массива не скроенных заказов может повторно в документе \myobject{План} системы \gofro запустить расчет раскроев и добавить новые раскрои к сохраненным ранее. Процесс планирования происходит до тех пор, пока не будет сформировано задание на объем, достаточный для загрузки производства на требуемый период.
\item	\planner при планировании  должен учитывать следующую информацию:
\begin{enumerate}
    \item наличие сырья;
    \item выпускаемую комбинацию сырья;
    \item количество заказа;
    \item количество слоев;
    \item профиль;
    \item возможности высекательного оборудования;
    \item размеры конструкции и ее особенности (дизайн);
    \item распределение и приоритет заказов по оборудованию с учетом технических особенности оборудования;
    \item отчет незавершенной продукции;
    \item ежемесячный план ППР оборудования;
    \item количества цветов печати.
\end{enumerate}
\item	\planner контролирует количество выпущенных заготовок по отчету \myform{ЗаготовкиПослеГА}.
%(форма \textit{Ф\_ЗаготовкиПослеГА}) 
% количество заказанных и полученных заготовок от поставщиков по отчету \$ОтчетПоЗаказамПоставщику.
\item	\planner планирует текущие потребности по сырью (Бумага и картон. Только для гофропроизводства) согласно процедуре \textbf{“Оперативное планирование сырья”}.
\item 	\planner в производственном задании в цех  указывает реальную номенклатуру сырья с учетом наличия на складах, чтобы \gaoperator не занимался задачей подбора сырья «на лету».
% \item Для указания того, что план сформирован и готов к выпуску, \planner должен «провести» его в СИСТЕМЕ.
%\item	\planner по сформированному и одобренному списку раскроев в системе \gofro при необходимости печатается производственное задание в цех из формы \#НепрерывныйПлан.
\item	\planner проводит документ  \myobject{План} в системе \gofro.

%%%%%%%%%%%%%%%%%%%%%%%%%%%%
%\item	\planner после нахождения раскроев производит планирование линий переработки под эти заказы на вкладке ПланированиеЛиний в документе  \#План в системе \gofro.
%\item	\planner открывает форму \#НепрерывныйПлан и выбирает из списка скроенные, проведенные, но еще не загруженные планы по гофроагрегату. В момент добавления новые задания встают в конец списка на гофроагрегаты и линии переработки.
%\item	\planner в форме \#НепрерывныйПлан  может оперативно редактировать сформированное задание, изменять порядок, количество и т.д. После изменения необходимо “произвести пересчет времени”, при котором система \gofro рассчитает плановое время выполнения каждого заказа.

%%%%%%%%%%%%%%%%%%%%%%%%%%%%
\end{enumerate}

\subsubsection{Сценарий ''Планирование выпуска продукции на линиях переработки''}
\label{bp:plan_3b}


\begin{enumerate}

\item	После построения раскроев для выдачи соответствующих заданий на линии переработки \planner в документе \myobject{План} в системе \gofro переходит на вкладку «Планирование линий»
\item	Планирование заданий на гофроагрегат и линий переработки осуществляется в одном плане.
\item	\planner в системе \gofro запускает механизм планирования линий переработки командой \myform{ЗапускПланированияЛиний}. Планирование может быть выполнено как с учетом времени выхода заказов с раскроев (время постановки на линию переработки не должно быть раньше времени выхода с гофроагрегата, к которому прибавлено время технологической отлежки), так и без учета времени выхода с гофроагрегата. Система \gofro производит автоматическую расстановку заказов по линиям переработки, отталкиваясь от возможных маршрутов изделия, которые указаны в справочнике  \myobject{ТехнологическаяКарта} системы \gofro.
\item	При расстановке заказов по линиям переработки система \gofro учитывает заведенные пользователями нерабочие временные промежутки (\myobject{ППР}). Кроме этого, система учитывает ограничение по оснастке по невозможности использовать одну и ту же оснастку на разных станках одновременно.
\item	После расчетов \planner в системе \gofro анализирует полученный результат. Для этого можно воспользоваться отчетом системы \gofro по простоям оборудования (Отчет \myform{ПростоиОборудования}). Если в отчете присутствуют простои (кроме переналадок оборудования), то план имеет «разрывы». Это может быть следствием учета времени  выхода с гофроагрегата (линия свободна, но нет других заказов, а подходящий заказ еще вырабатывается на гофроагрегате) или из-за ожидания освобождения оснастки используемой для выпуска на другом оборудовании. 
\item	Для устранения простоев или в целях ручной корректировки полученного задания \planner может в плане системы \gofro вручную отредактировать порядок следования заказов на линии переработки. Также \planner может вручную включить срочные заказы в план или перенести заказы с одного станка на другой.
\item	После интерактивного изменения порядка или списка заказов \planner в системе \gofro выполняет пересчет времени планируемого выпуска командой \myform{ПересчетВремени}, т. к. после ручного вмешательства оно становится некорректно.

\item	\planner проводит документ  \myobject{План} в системе \gofro.
\end{enumerate}



% \begin{comment} %НАЧАЛО КОММЕНТАРИЯ (ПОДРАЗДЕЛ СКРЫТ, 108-132)

\subsubsection{Сценарий ''Непрерывное планирование производства''}
\label{bp:plan_4}

\begin{enumerate}


\item Начальное планирование гофроагрегата и перерабатывающих линий осуществляется в оперативном плане и затем редактируется в форме \myobject{НепрерывныйПлан} системы \gofro. Задания на гофроагрегат и линии переработки в \myobject{НепрерывныйПлан} попадают в момент подгрузки  \planner оперативного плана. \planner открывает форму \myobject{НепрерывныйПлан} и выбирает из списка проведенный, но еще не загруженный оперативный план.
\item	Загруженные в \myobject{НепрерывныйПлан} задания на гофроагрегат и линии переработки встают в конец списка заданий для соответствующих станков.
\item	В \myobject{НепрерывныйПлан} \planner может оперативно редактировать сформированное задание, изменять порядок, перемещать заказы между станками, корректировать их объем и т.д. После изменения необходимо произвести <<пересчет времени>>, при котором система рассчитает плановое время выполнения каждого задания.
\item	Для устранения простоев или в целях ручной корректировки полученного задания \planner  может в \myobject{НепрерывныйПлан} в системе \gofro вручную отредактировать порядок следования заказов на линии переработки. При необходимости \planner может вручную изменить порядок производства для срочных заказов.
\item	После интерактивного изменения порядка или списка заказов \planner запускает процедуру пересчета времени планируемого выпуска, что позволяет при помощи механизмов системы  \gofro заметить потенциальные простои или прочие точки несогласованности, возникшие при редактировании заданий.
\item	В любой момент времени \myobject{НепрерывныйПлан} содержит актуальную информацию о списке запланированных заданий, которые доступны пользователю \planner в системе \gofro для редактирования.
\item	\master, \manager и любой другой пользователь системы \gofro получают доступ на просмотр актуального списка заданий в форме \myobject{НепрерывныйПлан}. У любого машиниста линии по кнопке дозагрузки видно список заданий на будущее.
\item	По мере выполнения заданий они будут «исчезать» из списка  \myobject{НепрерывныйПлан}  в системе \gofro.
\item	\planner в \myobject{НепрерывныйПлан} может посмотреть прогноз  остатков по заготовкам в цеху по всем заданиям в непрерывном плане, общий план загрузки, прогнозируемую загрузку заготовок.
\item	В \myobject{НепрерывныйПлан} доступна диаграмма с графическим представлением загруженности линий переработки, на которой наглядно видно «плотность» загрузки оборудования.
\item	В один момент времени с одним станком в \myobject{НепрерывныйПлан} может работать только один пользователь \planner.



\end{enumerate}

% \end{comment} 




\subsubsection{Сценарий ''Согласование и просмотр планов''}
\label{bp:plan_5}
\begin{enumerate}

%\item	\#НепрерывныйПлан доступен для просмотра пользователям системы \gofro. Дополнительная процедура согласования заданий не требуется.
%\item	Изменять задания в \#НепрерывныйПлан может только \planner.


%\item	\planner на базе плана в системе \gofro при необходимости создает документы выработки по станкам и гофроагрегату для последующего заполнения в цеху.
%\item Данные формы	\#НепрерывныйПлан доступны для просмотра пользователям системы \gofro.

\item	\planner распечатывает готовые документы \myobject{План} по форме <<Отчет раскроев>> и передает \master.

\end{enumerate}


\subsubsection{Сценарий ''Получение отчетов от производства''}
\label{bp:plan_6}


\begin{enumerate}


\item \planner может просматривать результаты выработки в соответствующих документах выработки производства (\myobject{ВыработкаГофроагрегата}, \myobject{ВыработкаПоПереработке}). При этом данные документы изменять могут только пользователи с особыми полномочиями.
\item \manager могут оперативно просматривать информацию о запланированных и выпущенных объемах на вкладке план-факт документа \myobject{Заказ}, отчетах по выработке, отчете \myform{ПортфельЗаказов}.
\end{enumerate}



\subsubsection{Сценарий ''Оперативное планирование сырья''}
\label{bp:plan_7}


\begin{enumerate}


\item	Система \gofro в документе \myobject{План} в форме редактирования раскроев после формирования списка раскроев \planner по команде \myform{ЗаполнитьКомпозиции} заполняет список композиций (на основании базовых композиций)  по найденным раскроям. Заполнение базовыми композициями (при их наличии) происходит только для раскроев с теми заказами, где в техкарте не была изначально заполнена композиция сырья (если композиция была в техкарте, то она автоматически попадает и в раскрой при планировании). 
\item \planner нажимает команду \myform{ОценкаНаличияСырья}, при этом система \gofro определяет текущие остатки по каждому виду сырья для раскроев по указанным композициям, определяет потребность и сравнивает с наличием на складе.
\item \planner для контроля потребностей в сырье использует отчет \myform{ОтчетПоСырью}, который вызывается в системе \gofro из формы редактирования раскроев документа \myobject{План}. В нем отображаются данные по суммарному требуемому для выполнения плана объему сырья и остатки сырья на складе (по всем складам). Если по каким-то позициям складских остатков не хватает на выполнение плана, то система \gofro предупреждает пользователя \planner, но не запрещает давать план в работу, так как сырье может поступить в ближайшее время и \planner об этом знает.
%\item Указанная выше оценка выполняется для новых раскроев на будущий период, а для определения текущей потребности в сырье (на ближайшую смену или сутки) \planner формирует отчет \$ПоСырью в \#НепрерывныйПлан в системе \gofro и передает на склад сырья \kladovshik.
\item \kladovshik 
% (\gaoperator работает ли в Штрих ??? ) 
получает от \planner потребность по сырью (бумаге и картону) и делает перемещение указанного сырья в заданном объеме на склад производства согласно процедуре “Перемещение ТМЦ” бизнес-процесса \textbf{“Учет ТМЦ”}. 
\end{enumerate}


% \subsubsection{Сценарий ''Планирование потребности в сырье (бумага, картон)''}
% \label{bp:plan_8}


% \begin{enumerate}

% \item	Расчет потребности в материалах на будущий период производится \purchase при помощи документа системы \gofro \#РасчетCырья на основании статистики использования сырья в предыдущие периоды и текущего списка активных заказов.
% \item	\purchase создает новый документ \#РасчетCырья и указывает
% \begin{enumerate}
%     \item диапазон дат для просмотра статистики использования сырья по проведенным планам гофроагрегата в системе \gofro;
% \item 	диапазон даты учета внесенных, но еще не выполненных заказов (например, все заказы с первого числа будущего месяца) – для учета уже реально поданных позиций \#Номенклатура;
% \item 	предполагаемый объем (м2) выпуска готовой продукции в рассчитываемом периоде.
% \end{enumerate}
% \item	\purchase для автоматического расчета потребности в сырье нажимает в системе \gofro кнопку \$Загрузить и система \gofro автоматически рассчитывает объем сырья с учетом критериев указанных пользователям. На основании статистики определяется список используемого сырья, а объем предполагаемого заказа устанавливается в пропорции объема использования за период статистики и таким образом, чтобы общий объем заказываемого сырья обеспечивал указанный предполагаемый объем выпуска готовой продукции.
% \item	\purchase вносит правки в рассчитанные объемы, добавляет/удаляет строки с сырьем.
% \item	\purchase формирует печатную форму по команде \$Печать. Печатная форма документа с разбивкой по форматам и номенклатурам сырья является основанием для заказа сырья у поставщиков.
% \item	\purchase на основании документа \#РасчетCырья создает документ \#ЗаказПоставщику.

% \end{enumerate}
