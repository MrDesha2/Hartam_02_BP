



\subsection{Описание бизнес-процесса «Продажа готовой продукции»}

\subsubsection{Сценарий ''Поиск потенциальных покупателей''}
\label{bp:sales_1}

\begin{enumerate}
    % \item \manager создает в системе \crm новый элемент   \myobject{Сделка}.  
    \item \manager создает в системе \gofro новый элемент справочника  \myobject{Контрагенты}, заполняет карточку нового контрагента.
    \item	В системе \gofro в справочнике  \myobject{Контрагенты} \manager заполняет дополнительную информацию по клиентам.
    \item	Новые записи вносятся (дублируются) в систему \unf в справочник \myobject{Контрагенты}.
    \item	Новые записи выгружаются автоматически в систему \unf в справочник \myobject{Контрагенты}.

 \item	После участия во встрече менеджер фиксирует в системе \gofro документ \blue{\#Встреча} с указанием контрагента, даты, места, представителей и описания протокола встречи
 \item	После проведения важного телефонного звонка \manager фиксирует в системе \gofro документ \blue{\#ТелефонныйЗвонок} с указанием даты, контрагента, абонента, предмета обсуждения.
% \item	Информация о клиенте не передается в \sbis пока для него в системе \gofro установлен статус «внутренний».
\end{enumerate}


\subsubsection{Сценарий ''Расчет стоимости изделия''}
\label{bp:sales_2}



\begin{enumerate}
\item 	\manager при получении от покупателя заявки на выпуск нового изделия (или пересчета цены по существующей позиции) должен определить требования покупателя по следующим параметрам:
\begin{itemize}
    \item  по качеству продукции на уровне марки бумаги /картона/ гофрокартона согласно принятым стандартам, а также размерам и условиям использования;
\item по объему поставок в разрезе одной заявки/месяца/года;
\item по условию вывоза продукции (самовывоз или доставка);
\item по способу упаковки и отправки товара (россыпью или на паллетах);
\item по условию оплаты (предоплата, отсрочка платежа);
\item ожидаемая клиентом цена за единицу продукции.
\end{itemize}

Также при необходимости \manager определяет требования покупателя по новому изделию:
\begin{itemize}
\item	Размеры изделия;
\item	Размеры заготовки;
\item	Марка, цвет, профиль;
\item	Файлы с дизайном печати и конструкцией ящика;
\item	Условия упаковки и отгрузки;
\item	Прочая информация по изделию.
\end{itemize}

\item	\manager на основании полученных данных выполняет расчет стоимости изделия в таблице \blue{@MSExcel} 
 для простых 
 %\todo{ФОРМА} 
 четырехклапанных коробов или листов гофрокартона. 
% При необходимости дополнительно могут использованы такие программы как \blue{@EasyPackMaker}, \blue{@AutoCAD}, \blue{@CorelDraw}.
\item	\manager в таблице \blue{@MSExcel} заполняет параметры изделия, полученные от заказчика.
\item	\manager в таблице \blue{@MSExcel}  заполняет композицию сырья и рассчитывает его стоимость.

% \item	\manager при выпуске более сложных изделий передает запрос на расчет по электронной почте для \tehnolog.

% Номенклатура будет в ГТ
% \item	\manager после согласования цены готового изделия создает в системе \erp при необходимости новую позицию справочника \#Номенклатура, при этом наименование новой позиции формирует из введенных параметров изделия в таблице MS Excel. 
% \item	\manager создает в системе \erp спецификацию цены на заданную дату.
% \item	\manager  в \erp печатает спецификацию.
\item	\manager после расчета цены оформляет в \blue{@MSWord} коммерческое предложение и отправляет клиенту.

\item Клиент в ответ высылает \manager  согласованное предложение с реквизитами для заключения договора. 

\item \manager готовит и согласует договор с заказчиком. Новый договор создается по шаблону договора.

% \item \manager готовит договор и спецификацию. В дальнейшем при создании заявок покупателя счет оформляется в системе \erp.

\item Клиент подписывает договор, спецификацию. %и оплачивает счет. 
% \item После подписания клиентом договора и спецификации ассистент менеджера ОМиП передает их в бумажном виде специалистам ОМиП для
% оформления пакета документов и создания заказа в производство.
\end{enumerate}



\subsubsection{Сценарий ''Определение требований покупателя''}
\label{bp:sales_3}

\begin{enumerate}
\item	После успешного согласования договора и условий поставки \manager по продажам гофропродукции создает в системе \gofro документ \myobject{ЗаявкаСпецификация}.



\item \manager по продажам гофропродукции в системе \gofro вводит требования по разработке нового изделия:

\begin{enumerate}
\item 	Дата;
\item 	Контрагент;
\item 	Наименование изделия;
\item 	Вид гофротары (тип изделия);
\item 	Размеры изделия и  заготовки;
\item 	Площадь;
\item 	Марка гофрокартона (поиск по наименованию в справочнике \myobject{Марки});
\item 	Профиль (поиск по наименованию в справочнике \myobject{Профиль}) ;
\item 	Количество цветов;
\item 	Паллетирование;
\item 	Габариты пачки;
\item 	Габариты поддона (поиск по наименованию в справочнике \myobject{Поддоны});
\item 	Количество изделий в пачке;
\item 	Количество пачек в ряду;
\item 	Количество рядов;
\item 	Количество изделий в паллете.
\item 	Прочая информация по изделию.
\end{enumerate}
\item \manager устанавливает статус «Новая» для документа \myobject{ЗаявкаСпецификация}.
\item 	Разработка технологической карты выполняется в соответствии с процессом \textbf{“Проектирование и разработка новой продукции”} процедура \textbf{“Получение запроса на разработку новой продукции”} (подробнее в пункте \ref{bp:pm_1}).
\item \manager	после изменения \tehnolog статуса  документа \myobject{ЗаявкаСпецификация} на “Выполнен” в системе \gofro открывает форму созданного элемента \myobject{ТехнологическаяКарта}, вызывает по команде \myform{ПечатьТК} отчет «Отчет для клиента» и отправляет из системы \gofro форму в формате pdf или распечатывает, подписывает их у клиента на бумажном носителе с указанием даты подписания и фамилий лиц с последующей передачей в архив.
\item 	При появлении замечаний от клиента \tehnolog вносит изменения в чертежи конструкции/дизайна и \manager повторно высылает  \myobject{ТехнологическаяКарта} на согласование клиенту.
\item 	После внесения изменений \tehnolog в чертежи конструкции/дизайна графики, подписанный и согласованный клиентом экземпляр чертежа конструкции/ дизайна графики может быть прикреплен в список файлов в системе \gofro в справочнике \myobject{ТехнологическаяКарта}. Служба планирования и другие службы получают последний верный вариант \myobject{ТехнологическаяКарта} в системе \gofro.
\item 	Любые изменения всех свойств элемента справочника \myobject{ТехнологическаяКарта} отслеживаются в системе \gofro с указанием пользователя и значения, которое было изменено.


\end{enumerate}




\subsubsection{Сценарий ''Заключение договора''}
\label{bp:sales_4}

\begin{enumerate}
\item \manager готовит и согласует договор и спецификацию к нему.
\item \manager создает элемент справочника \myobject{Договор} в системе \gofro, заполняет карточку договора.
\item	По регламенту справочник выгружается в систему \unf в справочник \myobject{Договор}.
% \item	\auditor сохраняет отсканированную копию подписанного договора в системе \erp в справочник  \myobject{Договор}.



\end{enumerate}


\subsubsection{Сценарий ''Передача заказа на изготовление''}
\label{bp:sales_5}


\begin{enumerate}
% \item \manager регистрирует в системе \gofro документ \myobject{Заявка} с указанием требований клиента по изготовлению продукции, сроков и объемов поставки. В помощь менеджеру система \gofro должна выводить по каждой номенклатуре текущие складские остатки готовой продукции по системе \erp (остатки загружаются в систему \gofro из системы \erp автоматически).
%\red{Нужно ли подгружать складские остатки готовой продукции из системы \erp, ГП в нашей систсеме?}
\item \manager  создает в системе \gofro документ \myobject{Заявка} с указанием требований клиента по изготовлению продукции, сроков и объемов поставки. В помощь менеджеру система \gofro должна выводить по каждой номенклатуре текущие складские остатки готовой продукции, учтенной в системе \gofro. Остатки ГП выводятся с учетом реализации, осуществляемой по системе \gofro. 


\item	При наличии предварительных заказов от клиентов большого объема с несколькими отгрузками в течение месяца \manager должен разбить такую заявку на несколько реальных производственных заказов на разные даты отгрузки.
\item	\manager выполняет расчет предварительной даты производства продукции в соответствии с бизнес-процессом \textbf{“Планирование выпуска готовой продукции”}, процедура \textbf{“Предварительное планирование производства”} (подробнее в пункте \ref{bp:plan}).
\item	\manager в системе \unf проверяет по контрагенту дебиторскую задолженность.
\item	После согласования сроков и объемов \manager в системе \erp создает счет на оплату при необходимости и выставляет его контрагенту. Затем \manager устанавливает в системе \gofro в документе \myobject{Заявка} статус «Одобрен к выпуску». 
\item	На основании одобренного документа \myobject{Заявка} в системе \gofro{} \manager создает производственные заказы (документ \myobject{Заказ}). 
\item	\planner в системе \gofro забирает заказы для планирования ежедневно. В план уходят проведенные документы \myobject{Заказ} со статусом «Одобрено к выпуску».
\item	\manager в системе \gofro контролирует сроки исполнения заказов в производстве опираясь на отчеты «Портфель заказов» и «Ожидаемый выпуск» в системе \gofro.
\item	Для того чтобы заказ больше не участвовал в планировании, ему необходимо установить соответствующий статус: «Выполнен», «Отменен». Ответственный \manager в системе \gofro должен периодически пользоваться формой подбора выполненных заказов для того, чтобы установить соответствующий статус. Для тех заказов, которые выпущены не в полном объеме, но при этом отгрузка остатков не требуется, тоже необходимо устанавливать статус «Выполнен».
\end{enumerate}


\subsubsection{Сценарий ''Контроль дебиторской задолженности''}
\label{bp:sales_6}


\begin{enumerate}
\item	\manager выполняет контроль дебиторской задолженности в системе \unf.
% Убрали 
% \item	В документе \blue{\#Заявка} в системе \gofro{} \manager нажимает команду \blue{\$ПолучитьЗадолженность}, cистема \gofro формирует запрос в системе \erp и получает задолженность по оплате по контрагенту и договору на дату документа. Величина задолженности показана на форме документа \blue{\#Заявка} системы  \gofro.
\end{enumerate}

\subsubsection{Синхронизация с КИС}
\label{bp:sales_integration}


\begin{enumerate}
\item Выгрузка справочных данных из \gofro в \erp производится автоматически.
\begin{enumerate}
\item 	Справочник  \myobject{Контрагенты} из \gofro загрузится в справочник  \myobject{Контрагенты} системы \unf.
\item	Справочник  \myobject{Договоры} из \gofro загрузится в справочник  \myobject{Договоры} системы \unf.
\item	Справочник  \myobject{Номенклатура} по материалам из \gofro загрузится в справочник  \myobject{Номенклатура} системы \unf.
% •	Журнал #Спецификация из @СБИС загрузится в регистр  #ЦеныНоменклатуры системы @ГТ.
\end{enumerate}

% \item Выгрузка справочных данных из \gofro в \erp   производится автоматически.

% \begin{enumerate}
% \item	Справочник  \myobject{Номенклатура} по готовой продукции по номенклатурной группе ''Готовая продукция'' из \gofro загрузится в справочник  \myobject{Номенклатура} системы \erp.

% В системе \gofro есть возможность разделения номенклатур по группам. Выгрузке из системы \gofro в справочник  \myobject{Номенклатура} системы \erp могут быть переданы определенные группы.
% \end{enumerate}

% \item	В документ \blue{\#Заявка} в систему \gofro по команде пользователя из системы \erp выгружается остаток дебиторской задолженности по параметрам: контрагенты, договоры, дата.
% \item	В документ \myobject{Заявка} в систему \gofro по команде пользователя из системы \erp выгружается остаток готовой продукции по параметрам: номенклатура, дата.
\item	Документ \blue{\#Заявка} после проведения в системе \gofro выгружается в систему \unf в документ \blue{\#ЗаказКлиента}.
\item	Документ \blue{\#Заказ} после проведения в системе \gofro выгружается в систему \unf в документ \blue{\#ЗаказВПроизводство}.
\item Данные по выработке из документов \myobject{ВыработкаГофроагрегата}, \myobject{ВыработкаПоПереработке} автоматически выгрузятся из \gofro в документы \myobject{ОтчетПроизводстваЗаСмену}{} системы \unf.
% \item	Документ \blue{\#Заявка} после проведения в системе \gofro выгружается в систему \erp в документ \blue{\#ЗаказКлиента}.



\end{enumerate}