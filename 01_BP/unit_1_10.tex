

\subsection{Описание бизнес-процесса «Управление качеством»}
\label{bp:quality}

\subsubsection{Сценарий ''Контроль качества продукции''}
\label{bp:quality_1}

\begin{enumerate}

\item \gaoperator самостоятельно проверяет качество заготовок;

\item \operator самостоятельно проверяет качество готовой продукции.

% \item \gaoperator приносит заготовку на проверку самостоятельно. 
% \item \laborant отдела контроля качества для образца заготовки измеряет толщину, влажность, продавливание, расслаивание и ECT.

% \item \operator первые коробки, выпущенные с производственной линии,  приносит \laborant отдела контроля качества для оценки качества.
% \item \laborant просматривает визуально, сравнивает выкраску и проверяет массу, толщину, влажность, продавливание, ЕСТ и ВСТ.
% \item \laborant результаты изменений заносит в журнал контроля качества. 
% \item \laborant отдела контроля качества по результатам формирует общий отчет в \blue{@MSExel}.
% \item \laborant в системе \erp печатает паспорт качества, заполняет и относит на склад готовой продукции.

\end{enumerate}