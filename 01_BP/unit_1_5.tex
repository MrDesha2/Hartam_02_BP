\newpage
\subsection{Описание бизнес-процесса «Учет ТМЦ»}
\label{bp:storage}

\subsubsection{Сценарий ''Прибытие автомобиля на склад''}
\label{bp:storage_1}


\begin{enumerate}

\item	\kladovshik получает от водителя приходные документы от поставщика.

\end{enumerate}




\subsubsection{Сценарий ''Предварительный контроль качества сырья''}
\label{bp:storage_2}


\begin{enumerate}

\item  Водитель погрузчика производит выгрузку сырья, \kladovshik осматривает его визуально. 
% При обнаружении дефектов сообщает кладовщику.

\item	\kladovshik 
% в присутствии \okk 
осматривает ТМЦ на соответствие заявленным характеристикам, сортности, целостности упаковки. При наличии отклонений составляет акт несоответствия.

\item Забракованное сырье водитель погрузчика должен отставить на склад брака.

\item \kladovshik при выявлении нарушения транспортировки составляет акт о повреждении и передает акт в бухгалтерию. Рулон принимается на склад в любом случае.
\end{enumerate}


\subsubsection{Сценарий ''Ввод новых элементов справочника Номенклатура''}
\label{bp:storage_4}



\begin{enumerate}
\item При появлении новых позиций ТМЦ \kladovshik  добавляет новые записи в справочник \myobject{Номенклатура} в системе \gofro. Справочник выгружаются по регламенту в систему \erp 
% и \erp 
% в справочник \myobject{Номенклатура} в систему \gofro.
\item Новые позиции по готовой продукции (гофропроизводство) добавляет \manager в системе \gofro. Справочник выгружается по регламенту из системы \gofro в справочник \myobject{Номенклатура} в систему \unf.
\end{enumerate}



\subsubsection{Сценарий ''Порулонная приемка сырья''. Только для гофропроизводства}
\label{bp:storage_5}

\begin{enumerate}
\item \kladovshik контролирует вес и номер рулона, после чего сравнивает с весом по номеру рулона в отвес-фактуре поставщика.

\item \kladovshik создает в системе \gofro документ ”Приходная накладная” (\myobject{Поступление ТМЦ})
% , в этом же документе распечатывает этикетки для каждого рулона 
и регистрирует поступление рулонов.
Регистрация поступления рулонов выполняется по каждому рулону с присвоением номера рулона по штрих-коду рулона.
\kladovshik по каждому рулону  вводит серию с указанием веса рулона, номера рулона. 

\item Контроль за остатками материалов осуществляется в системе \gofro по отчету ”Ведомость по партиям номенклатуры”.

\item \kladovshik проводит документ \myobject{ПриходнаяНакладная}.
\item \kladovshik передает сопроводительные документы по поступлению сырья в бухгалтерию.
% \item \kladovshik на основе приходных документов от поставщика в системе  \erp находит документ #ПоступлениеТМЦ, заполняет цены на сырье из сопроводительных документов и проводит документ.


\end{enumerate}


\subsubsection{Сценарий ''Поступление полуфабрикатов (покупных заготовок) и материалов, используемых на ГА (крахмала, едкого натра, буры)''. Только для гофропроизводства}
\label{bp:storage_51}

\begin{enumerate}
\item Первичный учет поступления ведется в системе \gofro.
\item \kladovshik контролирует количество и номенклатуру полуфабрикатов, сравнивая с данными документа поставщика.
\item \kladovshik  при поступлении в системе \gofro создает документ \myobject{ПоступлениеТМЦ}, при этом указывает поставщика и склад-получатель. 
\item \kladovshik  по каждой продукции системе \gofro указывает позицию из справочника \myobject{Номенклатура}, количество и цену.
\item \kladovshik  проводит документ \myobject{ПоступлениеТМЦ}.



\end{enumerate}



\subsubsection{Сценарий ''Перемещение рулонов''. Только для гофропроизводства}
\label{bp:storage_6}

\begin{enumerate}
\item  	Первичный учет перемещения рулонов ведется в системе \gofro.
\item 	\planner формирует план потребности по сырью на смену и передает \gaoperator.
% \item \kladovshik (\gaoperator) отслеживает в системе \gofro плановое время и количество подачи сырья \todo{Нет такого???}
%\item	\gaoperator в системе \stock печатает отчет <<ОтчетПоНеобходимомуСырью>> с разбивкой по времени и передает на склад \kladovshik.
\item	\kladovshik  при перемещении рулонов в системе \gofro создает документ \myobject{Перемещение},
% \todo{Проверить наличие такого документа???}, 
при этом указывает склад-отправитель и склад-получатель.
\item 	\kladovshik  указывает номера рулонов и позиции \#Номенклатура для перемещения в документе \myobject{ПеремещениеТМЦ}.
\item 	\kladovshik проводит документ \myobject{ПеремещениеТМЦ}.
\end{enumerate}



\subsubsection{Сценарий ''Перемещение ТМЦ''. Только для гофропроизводства}
\label{bp:storage_61}

\begin{enumerate}
\item  	Первичный учет перемещения прочих материалов ведется в системе \gofro документом \myobject{Перемещение}
% \item 	\planner формирует план потребности по сырью на смену и передает !МашинистуГА.
% \item	\gaoperator в системе \gofro печатает отчет ОтчетПоНеобходимомуСырью с разбивкой по времени и передает на склад \kladovshik.
\item	\kladovshik при перемещении ТМЦ  в системе \gofro создает документ \myobject{ПеремещениеТМЦ}, при этом указывает склад-отправитель и склад-получатель.
\item 	\kladovshik указывает позиции \myobject{Номенклатура} для перемещения в документе \myobject{ПеремещениеТМЦ} и количество.
\item 	\kladovshik проводит документ \myobject{ПеремещениеТМЦ} .
\end{enumerate}



% \subsubsection{Сценарий ''Списание рулонов бумаги и картона''}
% \label{bp:storage_7}

% \todo{Будем замарачиваться??? Им точно надо???}

% \begin{enumerate}
% \item  	Первичное списание рулонов бумаги и картона производится в системе \gofro на раскатах гофроагрегатов. 
% %\item  	Первичное списание рулонов бумаги и картона производится в системе \stock.
% %\item \gaoperator при учете сырья указывают его в бумажном бланке учета сырья.
% \item \gaoperator на раскате создает в системе \gofro документ \blue{\#СырьеДляВыработки}.
% \item \gaoperator на раскате в документе \blue{\#СырьеДляВыработки} 
% считывает сканером штрих-кода номер каждого рулона (на основании маркировочного ярлыка рулона) в колонку, соответствующую номеру слоя выпускаемой композиции (\textbf{Внимание! Номер слоя может не соответствовать номеру раската и вносить необходимо именно номер слоя}). После указания номера рулона система \gofro  автоматически заполняет вес рулона на основании складских остатков.
% \item \gaoperator на раскате при снятии рулона считывает номер рулона с ярлыка и указывает конечный диаметр (???) рулона в системе \gofro. Система \gofro рассчитывает расход сырья по рулону на основании разницы между начальным и конечным диаметрами рулона.
% % \todo[inline]{ВНИМАНИЕ!!! Оптисофт считает, что крайне нежелательно работать с диаметром. Необходимо взвешивать рулоны. При расчете по формуле постоянно будут происходить неверные вычисления оставшегося веса в рулоне. Если все-таки придется работать с диаметром, то Оптисофт ждет от Предприятия формулу по расчету веса на основании измеренного диаметра.}

% \item 	Если рулон был смотан полностью, \gaoperator на раскате в системе \gofro указывает нулевой конечный диаметр.

% \item 	Если при выпуске раскроев на гофроагрегате происходит замена слоя по сравнению с тем, что было указано по заданию, то \gaoperator должен указать в строке с раскроем в системе \gofro соответствующий новый слой.

% \item \gaoperator в конце смены в документе \blue{\#ВыработкаГофроагрегата} для табличной части сырья вызывает команду \blue{\$ЗаполнитьМатериалы},  система \gofro заполняет сырье с учетом номеров рулонов на основании данных списания материалов по документу \blue{\#СырьеДляВыработки}. При этом таблица материалов будет заполнена фактически распределенными рулонами на выпуск продукции с указанием номера заказа, номенклатуры, фактического веса рулона. 


% \item	Документ  \blue{\#ВыработкаГофроагрегата} с признаком «Проверено» автоматически выгружается в систему \stock  в документ \blue{\#Перемещение}. При этом табличная часть по фактически потребленным материалам будет заполнена на основании данных документа \blue{\#ВыработкаГофроагрегата} системs \gofro \todo{Требуется уточнение???}.

% % После установки \master в документе \#СписаниеТМЦ признака «Проверено» вносить изменения в документ сможет только пользователь, обладающий соответствующими правами.

% % \item \gaoperator в конце смены создает документ #СписаниеТМЦ на основании документа #ВыработкаГофроагрегата. При этом система @ГТ автоматически заполняет табличную часть документа #СписаниеТМЦ сырьем, указанным в документе #ВыработкаГофроагрегата. 
% % 8.	!Учетчик проводит документ #СписаниеТМЦ.

% \end{enumerate}



\subsubsection{Сценарий ''Списание рулонов бумаги и картона''}
\label{bp:storage_8}

\bigskip

\begin{enumerate}
\item \kladovshik создает документ \myobject{СписаниеТМЦ} в системе \gofro.
\item \kladovshik указывает позиции справочника \myobject{Номенклатура} для списания.
\item \kladovshik проводит документ \myobject{СписаниеТМЦ}.

\end{enumerate}


% \subsubsection{Сценарий ''Списание ТМЦ, используемых на операциях переработки (полуфабрикаты сторонних производителей, краска и др.) на основании отчетов производства''}
% \label{bp:storage_9}


% \begin{enumerate}
% \item Первичное списание прочих материалов при необходимости производится в системе \erp. 
% % 2.	!МастерЦГТ в документе #ВыработкаПоПереработке  заполняет в таблице “Материалы” в конце смены дополнительные материалы, использованные при производстве готовой продукции по факту.
% % 3.	!МастерЦГТ в конце смены создает документ #СписаниеТМЦ на основании документа #ВыработкаПоПереработке. При этом система @ГТ автоматически заполняет табличную часть документа #СписаниеТМЦ позициями материалов, указанными в документе #ВыработкаПоПереработке. 


% \end{enumerate}


\subsubsection{Сценарий ''Списание ТМЦ, используемых на операциях переработки (полуфабрикаты сторонних производителей, краска и др.) на основании отчетов производства''}
\label{bp:storage_10}


\begin{enumerate}
\item Первичное списание прочих материалов при необходимости производится в системе \gofro. 
\item \kladovshik списывает материалы в конце смены документом  \myobject{СписаниеТМЦ} 
% \todo{В бух нет такого документа. Уточнение, чем списывать и где ???}. 
% Передача материалов в кладовую выполняется в произвольном количестве, то есть это ненормируемые производственные затраты.
\end{enumerate}


\subsubsection{Сценарий ''Проведение инвентаризации''}
\label{bp:storage_11}


\begin{enumerate}
\item Проведение инвентаризации по готовой продукции производится в системе \gofro. 
\item Проведение инвентаризации по сырью (бумага и картон) производится в системе \gofro. \item	\kladovshik в системе \gofro создает документ \myobject{ИнвентаризацияТМЦ}.
\item	На предприятии создается инвентаризационная комиссия.
\item	По факту инвентаризации \kladovshik редактирует в системе \gofro в документе \myobject{ИнвентаризацияТМЦ} фактическое количество ТМЦ.
\item	По факту отклонения \kladovshik создает в системе документы \myobject{ОприходованиеТМЦ}, \myobject{ПоступлениеТМЦ} и \myobject{СписаниеТМЦ}.

% \item \kladovshik списывает материалы в конце смены документом \#ПередачаМатериаловВКладовую. 
% Передача материалов в кладовую выполняется в произвольном количестве, то есть это ненормируемые производственные затраты
\end{enumerate}


\subsubsection{Синхронизация с КИС}
\label{bp:storege_integration}

\begin{enumerate}
\item Выгрузка справочных данных из систем \gofro в систему \erp производится автоматически.

\begin{enumerate}
\item
Справочник  \myobject{Номенклатура} в части материалов из системы \gofro загрузится в Справочник  \myobject{Номенклатура} системы \erp.

\end{enumerate}

\item 	Выгрузка документов из системы \gofro в систему \erp производится автоматически по регламенту обмена.

% \begin{enumerate}
% \item
% Остатки материалов (бумага, картон) по всем складам на момент обмена из системы \gofro загрузится в документ  \myobject{ВводОстатков} системы \gofro.
% \end{enumerate}

% \item 	Выгрузка документов из системы \gofro в системы \erp и \stock производится автоматически по регламенту обмена.

 \begin{enumerate}
 \item Документ \myobject{ПоступлениеТМЦ} из системы \gofro  в документ \myobject{Поступление} системы  \unf.
 \item Документ \myobject{СписаниеТМЦ} из системы \gofro выгружается в документ \myobject{СписаниеТМЦ} системы  \unf.
 \item Документ \myobject{ПеремещениеТМЦ} из системы \gofro выгружается в документ \myobject{ПеремещениеТоваров} системы  \unf.
 \item Документ \myobject{ИнвентаризацияТМЦ} из системы \gofro выгружается в документ \myobject{ИнвентаризацияТоваров} системы  \unf.

% Потребление материалов (бумага, картон) по гофроагрегату за смену из системы \gofro загрузится в документ \blue{\#Перемещение} системы \stock.
 \end{enumerate}

\end{enumerate}