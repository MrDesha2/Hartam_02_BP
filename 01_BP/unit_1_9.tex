\subsection{Описание бизнес-процесса «Учет ремонтов и простоев»}
\label{bp:maintance}

% Модель бизнес-процессов представлена на Рис. \ref{pic:bp_production} Модель бизнес-процесса «Выпуск готовой продукции и полуфабрикатов».

% \textbf{Цель}


\subsubsection{Сценарий ''Регистрация останова и простоя оборудования''}
\label{bp:maintance_1}

\begin{enumerate}
\item	%Кроме тех случаев, когда простой на гофроагрегате или линии переработки фиксируется автоматически, может быть необходимость указать простой вручную. 
При останове оборудования \gaoperator и \operator  фиксирует в документе \myobject{ВыработкаГофроагрегата} и \myobject{ВыработкаПоПереработке} соответственно, фиксирует событие командой  \myform{УказатьПростой} и указывает один из признаков в журнале работы оборудования, с указанием времени начала останова. 
\item	После проведения ремонта или наладки оборудования \gaoperator указывает в системе \gofro в документе  \myobject{ВыработкаГофроагрегата}  событие о выполненных работах и время окончания останова.
\item	После проведения ремонта или наладки оборудования \operator указывает в системе \gofro в документе \myobject{ВыработкаПоПереработке} событие о выполненных работах и время окончания останова.

\end{enumerate}





\subsubsection{Сценарий ''Регистрация дефектов оборудования''}
\label{bp:maintance_2}

\begin{enumerate}
\item	При возникновении дефекта на  оборудовании \gaoperator фиксирует в документе \myobject{ВыработкаГофроагрегата}, \operator фиксирует в документе \myobject{ВыработкаПоПереработке} событие командой  \myform{УказатьДефект} и указывает один из дефектов в журнале дефектов оборудования. Если дефекта не оказалось в справочнике, дефект вписывается вручную. 

\end{enumerate}